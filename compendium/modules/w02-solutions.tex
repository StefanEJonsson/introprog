%!TEX encoding = UTF-8 Unicode
%!TEX root = ../compendium.tex

\ExerciseSolution{\ExeWeekTWO}

%Uppgift 1
\Task 

\Subtask värde: \code{Range(1,2,3,4,5,6,7,8,9)}, typ: \code{scala.collection.immutable.Range}

\Subtask värde: \code{Range(1,2,3,4,5,6,7,8,9,10)}, typ: \code{scala.collection.immutable.Range}

\Subtask värde: \code{Range(0,5,10,15,20,25,30,35,40,45)}, typ: \code{scala.collection.immutable.Range}

\Subtask värde: \code{10}, typ: \code{Int}

\Subtask värde: \code{Range(0,5,10,15,20,25,30,35,40,45,50)}, typ: \code{scala.collection.immutable.Range}

\Subtask värde: \code{11}, typ: \code{Int}

\Subtask värde: \code{Range(0,1,2,3,4,5,6,7,8,9)}, typ: \code{scala.collection.immutable.Range}

\Subtask värde: \code{Range(0,1,2,3,4,5,6,7,8,9)}, typ: \code{scala.collection.immutable.Range}

\Subtask värde: \code{Range(0,1,2,3,4,5,6,7,8,9)}, typ: \code{scala.collection.immutable.Range}

\Subtask värde: \code{Range(0,1,2,3,4,5,6,7,8,9,10)}, typ: \code{scala.collection.immutable.Range.Inclusive}

\Subtask värde: \code{Range(0,1,2,3,4,5,6,7,8,9,10)}, typ: \code{scala.collection.immutable.Range.Inclusive}

\Subtask värde: \code{Range(0,5,10,15,20,25,30,35,40,45)}, typ: \code{scala.collection.immutable.Range}

\Subtask värde: \code{Range(0,5,10,15,20,25,30,35,40,45,50)}, typ: \code{scala.collection.immutable.Range}

\Subtask värde: \code{11}, typ: \code{Int}

\Subtask värde: \code{500500}, typ: \code{Int}


%Uppgift 2
\Task 

\Subtask Ett objekt av typen \code{Array[String]} skapas med värdet \code{Array(hej, på, dej, !} och med namnet \code{xs}.

\Subtask Returnerar en sträng med värdet \code{hej}.

\Subtask Returnerar en sträng med värdet \code{!}.

\Subtask Ett exception genereras. Skriver ut \code{java.lang.ArrayIndexOutOfBoundsException: 4}.

\Subtask Returnerar en sträng med värdet \code{på dej}.

\Subtask Returnerar en sträng med värdet \code{hejpådej!}.

\Subtask Returnerar en sträng med värdet \code{hej på dej !}.

\Subtask Returnerar en sträng med värdet \code{(hej,på,dej,!)}.

\Subtask Returnerar en sträng med värdet \code{Array(hej,på,dej,!)}.

\Subtask Ett fel uppstår av typen \code{type mismatch}. Konsollen talar om för oss vad den fick, dvs värdet \code{42} av typen \code{Int}. Den talar även om för oss vad den ville ha, dvs något värde av typen \code{String}. Till sist skriver den ut vår kodrad och pekar ut felet.

\Subtask Det första elementet i \code{xs} ändras till värdet \code{42}. Därefter skrivs det första värdet i \code{xs} ut.

\Subtask Ett objekt av typen \code{Array[Int]} skapas med värdet \code{Array(42, 7, 3, 8} och med namnet \code{ys}.

\Subtask Returnerar summan av elementen i \code{ys}. Resultatet är \code{60}.

\Subtask Returnerar det minsta värdet i \code{ys}. Resultatet är \code{3}.

\Subtask Returnerar det största värdet i \code{ys}. Resultatet är \code{42}.

\Subtask Ett nytt värde av typen \code{Array[Int]} skapas med \code{10} stycken element, alla med värdet \code{42}.

\Subtask Returnerar summan av elementen i \code{zs}. Resultatet blir 420 (42 multiplicerat med 10).

\Subtask \code{r} tar upp 12 bytes. \code{a} tar upp ca 4 miljarder bytes.

%Uppgift 3
\Task 

\Subtask Ett objekt av typen \code{scala.collection.immutable.Vector[String]} initieras med värdet \code{Vector(hej, på dej, !)}.

\Subtask Returnerar det nollte elementet i \code{words}, dvs strängen \code{hej}.

\Subtask Returnerar det tredje elementet i \code{words}, dvs strängen \code{!}.

\Subtask Omvandlar vektorn till en Sträng.

\Subtask Samma som ovan, fast den här gången används mellanrum för att seperera elementen.

\Subtask Samma som ovan, fast den här gången sepereras elementen av kommatecken istället för mellanrum och dessutom börjar och slutar den resulterande strängen med parenteser.

\Subtask Samma som ovan, fast med ordet \code{Ord} tillagt i början av den resulterande strängen.

\Subtask Ett fel uppstår. Typen \code{Vector} är immutable. Dess element kan alltså inte bytas ut.

\Subtask En ny \code{Vector[Int]} skapas med värdet \code{Vector(42, 7, 3, 8)}. 

\Subtask Returnerar summan av vektorn \code{numbers}.

\Subtask Returnerar vektorns minsta element.

\Subtask Returnerar vektorns största element. 

\Subtask En ny vektor skapas innehållandes tiotusen 42or.

\Subtask Returnerar summan av vektorns element.

\Subtask Byta ut element.

%Uppgift 4
\Task 

\Subtask typ: \code{scala.collection.immutable.IndexedSeq[Int]}, värde: \code{Vector(1, 2, 3, 4, 5, 6, 7, 8, 9)}

\Subtask typ: \code{scala.collection.immutable.IndexedSeq[Int]}, värde: \code{Vector(1, 2, 3, 4, 5, 6, 7, 8, 9)}

\Subtask typ: \code{scala.collection.immutable.IndexedSeq[Int]}, värde: \code{Vector(2, 3, 4, 5, 6, 7, 8, 9, 10)}

\Subtask typ: \code{scala.collection.immutable.IndexedSeq[Int]}, värde: \code{Vector(1, 2, 3, 4, 5, 6, 7, 8, 9, 10)}

\Subtask typ: \code{scala.collection.immutable.IndexedSeq[Int]}, värde: \code{Vector(1, 2, 3, 4, 5, 6, 7, 8, 9, 10)}

\Subtask typ: \code{scala.collection.immutable.IndexedSeq[Int]}, värde: \code{Vector(2, 3, 4, 5, 6, 7, 8, 9, 10, 11)}

\Subtask typ: \code{Int}, värde: \code{Vector(65)}

\Subtask typ: \code{scala.collection.immutable.IndexedSeq[Int]}, värde: \code{Vector(0.0, 0.7071067811865475, 1.0, 0.7071067811865476, 1.2246467991473532E-16, -0.7071067811865475, -1.0, -0.7071067811865477)}

%Uppgift 5
\Task 

\Subtask typ: \code{scala.collection.immutable.IndexedSeq[Int]}, värde: \code{Vector(1, 2, 3, 4, 5, 6, 7, 8, 9, 10)}

\Subtask typ: \code{scala.collection.immutable.IndexedSeq[Int]}, värde: \code{Vector(1, 2, 3, 4, 5, 6, 7, 8, 9, 10)}

\Subtask typ: \code{scala.collection.immutable.IndexedSeq[Int]}, värde: \code{Vector(2, 4, 6, 8, 10, 12, 14, 16, 18, 20)}

\Subtask typ: \code{scala.collection.immutable.IndexedSeq[Int]}, värde: \code{Vector(2, 4, 6, 8, 10, 12, 14, 16, 18, 20)}

\Subtask typ: \code{scala.collection.immutable.Vector[Int]}, värde: En vector av tiotusen 85or (85 = 42 + 43).

%Uppgift 6
\Task 

\Subtask En \code{Range} skapas och dess element skrivs ut ett och ett.

\Subtask Samma sak händer.

\Subtask De tio första tio första jämna talen (noll ej inräknat) skrivs ut med ett "hej" framför.

\Subtask Talen 1 till 10 skrivs ut.

\Subtask Tiotusen slumptal mellan 0 och 1 genereras. Varje gång ett tal är större än 0.99 kommer det ett pling.

%Uppgift 7
\Task 

\Subtask Pseudokoden ska se ut ungefär såhär:

Skapa heltalsvariabel temp. 
Flytta värdet från x till temp. 
Flytta värdet från y till x. 
Flytta värdet från temp till y.

\Subtask
\begin{REPL}
scala> var (x, y) = (42, 43)
x: Int = 42
y: Int = 43
scala> var temp = x; x = y; y = temp;
temp: Int = 42
x: Int = 43
y: Int = 42
scala> println("x är " + x + ", y är " + y)
x är 43, y är 42
end{REPL}

%Uppgift 8
\Task 

\Subtask Skriver ut "hej skript".

\Subtask 

\Subtask Lägg till raden:
\code{println((2 to 1001).sum)} 
eller motsvarande.

\Subtask 

\Subtask 

%Uppgift 9
\Task 

\Subtask 

\Subtask 

\Subtask 

\Subtask 

%Uppgift 10
\Task 

\Subtask 

\Subtask 

\Subtask 

%Uppgift 11
\Task 

\Subtask 

\Subtask 

\Subtask 

\Subtask 

%Uppgift 12
\Task 

\Subtask 

\Subtask 

\Subtask 

\Subtask 

%Uppgift 13

\Subtask 

\Subtask 

\Subtask 

\Subtask 

\Subtask 

\Subtask 

\Subtask 

\Subtask 

\Subtask 

\Subtask 

\Subtask 

\Subtask 

%Uppgift 14
\Task 

\Subtask 

\Subtask 

\Subtask 

\Subtask 

%Uppgift 15
\Task 

\Subtask 

\Subtask 

%Uppgift 16
\Task 

\Subtask 

\Subtask 

\Subtask Lösningstext.
